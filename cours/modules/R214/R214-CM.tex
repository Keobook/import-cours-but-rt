\documentclass{report}

\begin{document}
  \title{Cours de R214}
  \author{Alexis Opolka}
  \date{\today}
  \maketitle

  \part{Cours magistraux}

  \chapter{Dérivation}
  \chapter{Limites et équivalents}
  \chapter{Intégration}

    Primitive:

    \underline{Def:} Soit f(x) et F(x) deux fonctions.

    On a:

    f(x) est \underline{la} dérivée de F(x)
    F(x) est \underline{une} primitive de f(x)

    [Caution] F(x) + 2, F(x) - 5, ...
    sont aussi des primitives de f(x)

    Primitive de $\frac{1}{x} \ln{|x|}$

    \underline{Exemple 1:}

    a/ $f(x) = 4x + 3 => F(x) = 2x^2 + 3x + k, k \in \Re$ \\
    b/ $f(x) = \frac{1}{2x - 3} => F(x) = \frac{1}{2} \ln{2x-3} + k$
      car ($\ln{2x-3})' =\frac{2}{2x-3}$ \footnote{($\ln{u(x)})' = \frac{u'(x)}{u(x)}$}

    \underline{Etape 1 :} Réduction
    \underline{Exemple 2:}

    f(x) = $\frac{x^3-1}{x^2-2x+1} = \frac{(x-1)(x^2 +x +1)}{(x-1)^2} = $


    Division euclidiènne

    x^3 +1 / x-1
    x^2 => x^3 -x^2 => x^2 + 1
    x => x^2 - x => x + 1
    1 => x - 1 => 2
\end{document}
